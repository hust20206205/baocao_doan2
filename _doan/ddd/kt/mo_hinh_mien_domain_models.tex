Để tạo một phần mềm tốt, chúng ta cần phải hiểu rõ về phần mềm đó. Trong thiết kế hướng miền để có thể hiểu miền nhanh, chúng ta cần tạo ra các mô hình miền. Mô hình miền (Domain Models) là kiến thức có tổ chức và có cấu trúc về miền phù hợp để giải quyết vấn đề kinh doanh. Mục tiêu của mô hình miền là cung cấp rõ ràng, ngắn gọn và chính xác về miền làm cơ sở để hệ thống giải quyết vấn đề kinh doanh.

% \begin{example} Trong đồ án này, mô hình miền của em bao gồm các yêu cầu nghiệp vụ và các sơ đồ: UML Use Case Diagrams, UML Class Diagrams,\dots kĩ thuật ở phần sau

% \end{example}