Thiết kế hướng miền được Eric Evans giới thiệu trong cuốn sách               \textit{"Domain Driven Design: Tackling Complexity in the Heart of Software"}. \emph{Thiết kế hướng miền (Domain Driven Design) }  là một tư tưởng, một hướng tiếp cận   thiết kế phần mềm tập trung vào việc hiểu rõ và mô hình hóa lĩnh vực kinh doanh của một tổ chức.          Thiết kế hướng miền nhấn mạnh việc sử dụng lĩnh vực nghiệp vụ kinh doanh để thảo luận và đề xuất giải pháp đáp ứng nhu cầu. Vì để tạo một phần mềm tốt, chúng ta cần phải hiểu rõ về chính phần mềm đó.

Với nhiều phần mềm phân tích thiết kế  không tốt, phần xử lý các công việc không liên quan đến vấn đề nghiệp vụ kinh doanh như truy cập tập tin, hạ tầng mạng, cơ sở dữ liệu,... được lập trình trong đối tượng nghiệp vụ kinh doanh. Cách này giúp tốc độ hoàn thiện dự án nhanh. Tuy nhiên, cách này làm   dự án  bị mất đi tính hướng đối tượng khó thay  thay đổi, bảo trì, mở rộng hệ thống,\dots        Thiết kế hướng miền  cung cấp một cách để tổ chức mã và làm cho nó dễ bảo trì hơn,  dễ dàng thích ứng với các yêu cầu thay đổi.  

Trong kiến trúc vi dịch vụ, thiết kế hướng miền giúp đảm bảo rằng mỗi dịch vụ được thiết kế phản ánh một phần cụ thể của lĩnh vực kinh doanh. Mỗi dịch vụ được quản lí bởi một nhóm  phát triển được hỗ trợ bởi các chuyên gia ngành.      Trong thiết kế hướng miền, \emph{chuyên gia ngành (Domain Expert)} là người có kiến thức và hiểu biết sâu sắc về vấn đề đang được hệ thống phần mềm giải quyết. Chuyên gia ngành thể hiện chính xác vấn đề kinh doanh, đóng vai trò là nguồn thông tin cho nhóm phát triển.





Hệ thống phần mềm được tạo ra để xử lý sự phức tạp trong cuộc sống hiện đại. Việc phát triển hệ thống liên kết chặt chẽ với một số khía cạnh cụ thể trong cuộc sống của chúng ta.  Trong thiết kế hướng miền,      \emph{miền (Domain)} đề cập đến phạm vi kiến thức và vấn đề cụ thể mà hệ thống xử lý. Xét  miền  theo góc độ kinh doanh và góc độ hệ thống:

\begin{itemize}

    \item Về góc độ kinh doanh: Miền đại diện cho một lĩnh vực hoặc ngành mà doanh nghiệp hoạt động.

    \item Về góc độ hệ thống: Miền có thể coi là đại diện cho không gian vấn đề của hệ thống.

\end{itemize}


\begin{example} \emph{Trong đồ án này, miền được xác định là bài toán giải pháp hóa đơn điện tử.}

\end{example}