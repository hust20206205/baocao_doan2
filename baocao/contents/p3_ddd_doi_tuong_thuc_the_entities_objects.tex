Trong  thiết kế hướng miền, đối tượng thực thể (Entities Objects)  là        đối tượng miền   có định danh riêng duy nhất. Định danh này được giữ nguyên xuyên suốt trạng thái hoạt động của hệ thống phần mềm. Một đối tượng       thực thể     được phân biệt với các đối tượng thực thể  khác dựa trên nhận dạng duy nhất.

Danh tính của một thực thể phải là duy nhất và bất biến từ đó đạt được nhất quán về dữ liệu    và thể hiện mối quan hệ với các thực thể   khác.


Các thực thể   là các đối tượng quan trọng nhất trong mô hình miền     bao gồm các thuộc tính và hành vi miền được xác định rõ ràng.  Khi các hoạt động  hành vi được thực hiện đối với thực thể sẽ dẫn đến sự thay đổi trạng thái của thực thể.    Các thực thể được lưu trữ lâu dài thể hiện trạng thái hiện tại của thực thể. 


\begin{example}     Trong trường hợp CSDL quan hệ,   bảng            thể hiện  thực thể và được  định danh duy nhất bằng cột khóa chính.
    
\end{example}



