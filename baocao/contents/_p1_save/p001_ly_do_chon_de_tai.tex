Trong quá trình hoạt động kinh doanh, doanh nghiệp có nhu cầu chuyển đổi mô hình kinh doanh linh hoạt để có thể tồn tại và phát triển khi thị trường thay đổi. Từ đó, đáp ứng nhu cầu của khách hàng, mang lại ưu thế cạnh tranh so với các đối thủ.

Trong những năm gần đây, việc áp dụng kiến trúc vi dịch vụ ngày càng phổ biến, đem lại nhiều lợi ích như tách các nghiệp vụ kinh doanh thành các dịch vụ nhỏ độc lập, tăng tính linh hoạt và khả năng chống chịu sự cố của hệ thống.

Kiến trúc vi dịch vụ hỗ trợ doanh nghiệp chuyển đổi nhanh chóng để đáp ứng nhu cầu của mô hình kinh doanh và mong đợi của khách hàng. Tuy nhiên, để xây dựng được kiến trúc vi dịch vụ tốt, cần phải tạo ra các dịch vụ nhỏ phù hợp và duy trì tính độc lập. Trong đồ án này, em sử dụng thiết kế hướng miền để phân tích và xây dựng kiến trúc vi dịch vụ.

Theo quy định của Nghị định 123/2020/NĐ - CP, tất cả các doanh nghiệp, tổ chức và hộ kinh doanh đều bắt buộc phải sử dụng hóa điện tử. Vì vậy, nhu cầu sử dụng và xử lý hóa đơn điện tử trở nên rất lớn. Do đó trong đồ án này, em chọn chủ đề \emph{"Sử dụng thiết kế hướng miền xây dựng kiến trúc vi dịch vụ cho bài toán hóa đơn điện tử"}. Chủ đề này là một xu hướng quan trọng trong phát triển phần mềm và mang lại nhiều lợi ích trong việc cải thiện quá trình quản lý hóa đơn điện tử.