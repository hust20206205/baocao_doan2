% Tổng hợp  (Aggregates)


Tổng hợp là đối tượng kinh doanh trung tâm trong Bối cảnh bị giới hạn của chúng ta và xác định phạm vi nhất quán trong bối cảnh bị giới hạn đó.

Tổng hợp = Mã định danh chính của Bối cảnh bị giới hạn của chúng ta










**Quản lý vòng đời của các đối tượng miền**

Việc quản lý vòng đời các đối tượng trong miền không hề đơn giản, nếu như làm không đúng sẽ có thể gây ảnh hưởng đến việc mô hình hóa miền.

**Mẫu tổng hợp (Aggregate)**

<!--Tính tương đồng (Aggregate)-->

Mẫu tổng hợp là một nhóm các thực thể và đối tượng giá trị được xem như một tổng thể thống nhất từ ​​góc độ dữ liệu và khái niệm miền.

<!--Hãy để tôi giải thích điều này bằng một minh họa.-->

Một tập hợp bao gồm một nhóm tổng hợp còn được gọi là thực thể gốc.

Thực thể gốc này có một danh tính duy nhất từ ​​phối cảnh miền.

Phần thứ hai của tập hợp là cụm, được hình thành bởi ranh giới của tập hợp.

Trong ranh giới này, có thể không có hoặc nhiều thực thể tổng hợp và đối tượng giá trị. Các đối tượng trong cụm này hoặc đối tượng trong ranh giới được gọi là đối tượng bên trong hoặc đối tượng con.

![](image-4.png)

Aggregate phải cung cấp các giao diện để vận hành trên các đối tượng bên trong.

đảm bảo rằng tất cả hành vi cần thiết để vận hành trên đối tượng bên trong được hiển thị dưới dạng các hàm của đối tượng gốc tổng hợp.

![](image-5.png)




<!-- % Aggregate -->

<!-- Aggregate: https://ddd-practitioners.com/home/glossary/aggregate/ -->

<!-- [[Aggregate]] A cluster of associated objects that are treated as a unit for the purpose of data changes. External references are restricted to one member of the AGGREGATE, designated as the root. A set of consistency rules applies within the AGGREGATE’S boundaries. -->

<!-- State Stored Aggregates : https://ddd-practitioners.com/state-stored-aggregate -->



Chuyển đến nội dung
Đối với người hành nghề bởi người hành nghề
Tìm kiếm
Thiết kế hướng miền: Hướng dẫn dành cho người thực hành
Câu hỏi thường gặp
Bảng chú giải
Về chúng tôi
Cuốn sách của chúng tôi!
Trang chủTrang chủBảng chú giảiTổng hợp
Tổng hợp
Trong thiết kế hướng miền (DDD), tập hợp là một nhóm các đối tượng miền được coi là một đơn vị duy nhất. Một tập hợp xác định một ranh giới nhất quán, có nghĩa là tất cả các thay đổi đối với các đối tượng trong tập hợp phải được thực hiện theo cách duy trì tính nhất quán và tính toàn vẹn của các đối tượng.

Một tập hợp có một gốc, là đối tượng chính trong tập hợp và một hoặc nhiều đối tượng con. Root chịu trách nhiệm duy trì tính nhất quán và toàn vẹn của tập hợp và nó kiểm soát quyền truy cập vào các đối tượng con.

Ví dụ: trong miền thương mại điện tử, tổng hợp có thể là một đơn hàng, bao gồm đối tượng tiêu đề đơn hàng và một hoặc nhiều đối tượng chi tiết đơn hàng. Đối tượng tiêu đề đơn hàng sẽ là gốc của tổng hợp và nó sẽ chịu trách nhiệm duy trì tính nhất quán của đơn hàng, chẳng hạn như đảm bảo rằng tổng đơn hàng là chính xác và đơn hàng ở trạng thái hợp lệ.

Một tập hợp phải có kích thước nhỏ, nhất quán và phải được thiết kế càng nhỏ càng tốt, điều này nhằm đảm bảo rằng độ phức tạp của tập hợp có thể quản lý được và có thể dễ dàng duy trì tính nhất quán.

Trong DDD, các tập hợp được sử dụng để mô hình hóa các ranh giới nhất quán trong miền và để đảm bảo rằng trạng thái của miền luôn nhất quán. Chúng cũng được sử dụng để kiểm soát quyền truy cập vào các đối tượng con và cung cấp cách thực hiện các hoạt động liên quan đến nhiều đối tượng trong tổng thể.

Khi thiết kế một tập hợp, điều quan trọng là phải xác định ranh giới nhất quán, chọn đối tượng gốc thích hợp và đảm bảo rằng tập hợp đó nhỏ và nhất quán.

Tổng hợp so với các đối tượng trong OOP
Trong lập trình hướng đối tượng (OOP), một đối tượng là một đơn vị hành vi và dữ liệu độc lập đại diện cho một thể hiện duy nhất của một lớp. Các đối tượng có trạng thái và hành vi, đồng thời chúng có thể tương tác với các đối tượng khác bằng cách gửi và nhận tin nhắn.

Trong thiết kế hướng miền (DDD), một tập hợp cũng là một nhóm đối tượng, nhưng nó có vai trò và mục đích cụ thể. Một tập hợp xác định một ranh giới nhất quán, có nghĩa là tất cả các thay đổi đối với các đối tượng trong tập hợp phải được thực hiện theo cách duy trì tính nhất quán và tính toàn vẹn của các đối tượng.

Một tập hợp có một gốc, là đối tượng chính trong tập hợp đó và một hoặc nhiều đối tượng con. Root chịu trách nhiệm duy trì tính nhất quán và toàn vẹn của tập hợp và nó kiểm soát quyền truy cập vào các đối tượng con. Một tập hợp phải có kích thước nhỏ, nhất quán và phải được thiết kế càng nhỏ càng tốt, điều này nhằm đảm bảo rằng độ phức tạp của tập hợp có thể quản lý được và có thể dễ dàng duy trì tính nhất quán.

Ngược lại với OOP, trong đó các đối tượng là các đơn vị hành vi và dữ liệu độc lập, trong DDD, tổng hợp là một mẫu được sử dụng để tạo ranh giới nhất quán và để đảm bảo tính toàn vẹn của miền. Nó giúp quản lý độ phức tạp của miền bằng cách chia nó thành các phần nhỏ hơn, dễ quản lý hơn.

Tóm lại, đối tượng của OOP là một đơn vị hành vi và dữ liệu độc lập, trong khi đó, tổng hợp của DDD là một mẫu.

Tổng hợp được lưu trữ theo trạng thái và có nguồn gốc từ sự kiện
Các tập hợp có thể được mô hình hóa theo hai cách, bằng cách lưu trữ trạng thái hiện tại của chúng hoặc bằng cách xây dựng trạng thái từ các sự kiện miền trong quá khứ. Các tập hợp tổng hợp được lưu trữ trạng thái và các tập hợp tổng hợp có nguồn gốc sự kiện đều là những cách để mô hình hóa các đối tượng miền trong ngữ cảnh thiết kế hướng miền (DDD).

Trong các tập hợp được lưu trữ trạng thái, trạng thái hiện tại của tập hợp được lưu trữ trong bộ nhớ hoặc trong cơ sở dữ liệu và các thay đổi đối với trạng thái được thực hiện bằng cách gọi các phương thức trên tập hợp. Cách tiếp cận này quen thuộc hơn với các nhà phát triển đã quen với lập trình hướng đối tượng truyền thống vì nó phản ánh chặt chẽ cách các đối tượng hoạt động trong hầu hết các ngôn ngữ lập trình.

Ngược lại, các tập hợp tổng hợp có nguồn gốc từ sự kiện lưu trữ toàn bộ lịch sử của tất cả các thay đổi đối với tập hợp dưới dạng một chuỗi các sự kiện. Để thay đổi trạng thái tổng hợp, một sự kiện mới sẽ được thêm vào luồng sự kiện, thay vì sửa đổi trực tiếp trạng thái. Cách tiếp cận này cho phép linh hoạt hơn trong việc xử lý đồng thời và khả năng phục hồi, đồng thời nó cũng cho phép kiểm tra và gỡ lỗi hệ thống tốt hơn.

Tìm nguồn cung ứng sự kiện phức tạp hơn so với phương pháp lưu trữ trạng thái, nhưng được cho là mang lại nhiều lợi ích hơn về khả năng mở rộng, khả năng chịu lỗi và khả năng kiểm tra.

Tổng hợp so với gốc tổng hợp so với thực thể
Tổng hợp là một cụm các đối tượng liên quan được coi là một đơn vị đối với các thay đổi dữ liệu. Gốc tổng hợp là đối tượng chính trong một tổng hợp, hoạt động như một người gác cổng để đảm bảo tính nhất quán và tính toàn vẹn của tổng thể. Thực thể là một đối tượng duy nhất trong một miền đại diện cho một đối tượng trong thế giới thực và có mã định danh duy nhất.

Sự khác biệt giữa một tổng hợp và một gốc tổng hợp là một tổng hợp là một nhóm các đối tượng được coi là một đơn vị, trong khi gốc tổng hợp là đối tượng chính đóng vai trò là điểm vào của tổng hợp và chịu trách nhiệm thực thi các bất biến của tổng hợp. . Mặt khác, một thực thể là một đối tượng duy nhất trong một miền đại diện cho một đối tượng trong thế giới thực và có thể là một phần của tổng thể.

Ví dụ, hãy xem xét một ứng dụng ngân hàng. Tổng hợp trong miền này có thể là tài khoản khách hàng, bao gồm các đối tượng liên quan như thông tin khách hàng, giao dịch và số dư tài khoản của họ. Gốc tổng hợp trong trường hợp này sẽ là đối tượng tài khoản khách hàng, chịu trách nhiệm thực thi các quy tắc xung quanh số dư (ví dụ: nó không bao giờ được âm). Khi các thay đổi được thực hiện đối với tài khoản, chúng được thực hiện thông qua thư mục gốc tổng hợp, đảm bảo rằng các quy tắc được duy trì.


Thể loại

Phân tích
điều cơ bản
ddd
thiết kế
câu hỏi thường gặp
Khả năng lãnh đạo
hoa văn
Blog tại WordPress.com.









Chuyển đến nội dung
Đối với người hành nghề bởi người hành nghề
Tìm kiếm
Thiết kế hướng miền: Hướng dẫn dành cho người thực hành
Câu hỏi thường gặp
Bảng chú giải
Về chúng tôi
Cuốn sách của chúng tôi!
Trang chủTrang chủBảng chú giảiTổng hợpTổng hợp được lưu trữ nhà nước
Tổng hợp được lưu trữ nhà nước
Trong Thiết kế hướng miền (DDD), tập hợp là một cụm các đối tượng liên quan được coi là một đơn vị. Có hai loại tổng hợp chính: lưu trữ trạng thái và lấy nguồn sự kiện.

Tổng hợp được lưu trữ trạng thái là tổng hợp dựa trên trạng thái hiện tại của các đối tượng của nó. Trạng thái hiện tại được lưu trữ trong cơ sở dữ liệu và việc cập nhật trạng thái được thực hiện bằng cách sửa đổi trực tiếp các thuộc tính của đối tượng. Tổng hợp được lưu trữ trạng thái thường được triển khai bằng cách sử dụng trình ánh xạ quan hệ đối tượng (ORM) hoặc công nghệ truy cập dữ liệu tương tự.

Ngược lại, tổng hợp nguồn sự kiện dựa trên một chuỗi các sự kiện đã xảy ra trong hệ thống. Các sự kiện được lưu trữ trong kho sự kiện và trạng thái hiện tại của tổng hợp được lấy bằng cách phát lại các sự kiện. Mỗi sự kiện thể hiện một sự thay đổi về trạng thái của tổng hợp và các sự kiện này được sử dụng để xây dựng lại trạng thái hiện tại của tổng hợp.

Sự khác biệt chính giữa hai loại uẩn là cách chúng lưu trữ trạng thái của chúng. Các tập hợp tổng hợp được lưu trữ trạng thái lưu trữ trạng thái của chúng trong cơ sở dữ liệu, trong khi các tập hợp tổng hợp có nguồn gốc sự kiện lưu trữ trạng thái của chúng theo một chuỗi sự kiện. Các tập hợp tổng hợp được lưu trữ trạng thái thường dễ triển khai hơn và mang lại hiệu suất tốt hơn cho các trường hợp sử dụng đơn giản. Tuy nhiên, các thông tin tổng hợp có nguồn gốc từ sự kiện có khả năng thay đổi linh hoạt hơn và có thể cung cấp bản kiểm tra tốt hơn về các thay đổi được thực hiện đối với hệ thống theo thời gian.

Ví dụ
Giả sử chúng ta có một ứng dụng thương mại điện tử và muốn lập mô hình giỏ hàng. Chúng ta có thể tạo một Carttổng hợp duy trì trạng thái của giỏ hàng, bao gồm các mặt hàng trong giỏ hàng và tổng giá. Tổng Carthợp có thể có các phương pháp để thêm và xóa các mặt hàng cũng như để tính tổng giá.

1
2
3
4
5
6
7
số 8
9
10
11
12
13
14
15
16
17
18
19
20
21
22
23
24
25
26
27
28
29
30
31
32
33
34
35
36
37
public class Cart {
  private Map<String, Integer> items = new HashMap<>();
  private BigDecimal total = BigDecimal.ZERO;
 
  public void addItem(String productId, int quantity, BigDecimal price) {
    if (items.containsKey(productId)) {
      int existingQuantity = items.get(productId);
      items.put(productId, existingQuantity + quantity);
    } else {
      items.put(productId, quantity);
    }
    total = total.add(price.multiply(BigDecimal.valueOf(quantity)));
  }
 
  public void removeItem(String productId, int quantity, BigDecimal price) {
    if (!items.containsKey(productId)) {
      return;
    }
 
    int existingQuantity = items.get(productId);
    if (existingQuantity < quantity) {
      return;
    }
 
    int newQuantity = existingQuantity - quantity;
    if (newQuantity == 0) {
      items.remove(productId);
    } else {
      items.put(productId, newQuantity);
    }
    total = total.subtract(price.multiply(BigDecimal.valueOf(quantity)));
  }
 
  public BigDecimal getTotal() {
    return total;
  }
}
Tổng hợp này Cartlà tổng hợp được lưu trữ trạng thái vì nó duy trì trạng thái của nó trong bộ nhớ. Khi một thao tác được thực hiện trên Cart, chẳng hạn như thêm hoặc xóa một mục, trạng thái sẽ được cập nhật trong bộ nhớ. Không có sự tồn tại dai dẳng của các sự kiện dẫn đến tình trạng hiện tại.


Thể loại

Phân tích
điều cơ bản
ddd
thiết kế
câu hỏi thường gặp
Khả năng lãnh đạo
hoa văn
Blog tại WordPress.com.