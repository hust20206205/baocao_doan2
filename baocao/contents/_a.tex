% <!-- Infrastructure Service : https://ddd-practitioners.com/infrastructure-service -->
% <!-- Application Service : https://ddd-practitioners.com/application-service -->
% <!--  : https://ddd-practitioners.com/layered-architecture -->
% <!-- Module : https://ddd-practitioners.com/?page_id=618 -->


Khi phát triển ứng dụng phần mềm, một phần lớn thành phần không liên quan trực tiếp đến nghiệp vụ, nhưng chúng là một phần của hạ tầng. Ví dụ như truy cập CSDL, hạ tầng mạng, ... Trong một ứng dụng hướng đối tượng thuần túy, các đoạn mã lại được nhúng vào trong các hành vi của các đối tượng nghiệp vụ vì nó là cách dễ và nhanh chóng. Tuy nhiên, việc trộn lẫn các đoạn mã liên quan đến nghiệp vụ có thể làm cho việc sửa đổi  khó khăn, kém linh hoạt.

Theo thiết kế hướng miền có 4 lớp:
\begin{itemize}
    \item Lớp giao diện người dùng (User Interface)
    \item Lớp ứng dụng (Application Layer)
    \item Lớp miền (Domain Layer)
    \item Lớp hạ tầng (Infrastructure Layer)
\end{itemize}


\subsubsection{Lớp giao diện người dùng (User Interface)}
Chịu trách nhiệm trình bày thông tin tới người   dùng và  tương tác với  người dùng.
\subsubsection{Lớp ứng dụng (Application Layer)}

\subsubsection{Lớp miền (Domain Layer)}
Là lớp quan trọng theo  thiết kế hướng   miền,  lớp miền chứa các đối tượng nghiệp vụ và các quy tắc nghiệp vụ kinh doanh. 
\subsubsection{Lớp hạ tầng (Infrastructure Layer)} 

Lớp này đóng vai trò  hỗ trợ cho tất cả các lớp còn lại. Nó cung cấp thông tin liên lạc giữa các lớp.



 


 
% Kiến trúc lớp là một mẫu thiết kế phần mềm phân chia các mối quan tâm khác nhau của hệ thống phần mềm thành các lớp hoặc thành phần khác nhau, trong đó mỗi lớp cung cấp một tập hợp dịch vụ cụ thể cho lớp phía trên nó. Sự tách biệt các mối quan tâm này giúp duy trì tính mô-đun và khả năng mở rộng của hệ thống, cũng như giảm độ phức tạp của thiết kế tổng thể.

% Trong Thiết kế hướng miền (DDD), kiến ​​trúc phân lớp thường được sử dụng để tổ chức miền giải pháp thành các phần khác nhau. Ví dụ: nó có thể bao gồm lớp trình bày, lớp ứng dụng, lớp miền và lớp cơ sở hạ tầng. Lớp trình bày xử lý sự tương tác của người dùng, lớp ứng dụng triển khai các trường hợp sử dụng và quy trình làm việc, lớp miền thể hiện các quy tắc và thực thể nghiệp vụ, còn lớp cơ sở hạ tầng cung cấp các dịch vụ kỹ thuật như lưu giữ lâu dài, nhắn tin và bảo mật.

% Việc sử dụng kiến ​​trúc phân lớp trong DDD giúp đảm bảo rằng các phần khác nhau của hệ thống được phân tách rõ ràng và các tương tác giữa chúng được xác định rõ ràng. Điều này giúp việc hiểu và bảo trì hệ thống dễ dàng hơn cũng như mở rộng hệ thống trong tương lai khi có yêu cầu mới.

% Ví dụ
% Một ví dụ về kiến ​​trúc phân lớp có thể là một ứng dụng web bao gồm các lớp sau:

% Lớp trình bày : Lớp này có nhiệm vụ trình bày giao diện người dùng cho người dùng. Nó bao gồm các thành phần như bộ điều khiển, dạng xem và mẫu.
% Lớp ứng dụng : Lớp này chịu trách nhiệm triển khai các trường hợp sử dụng hoặc logic nghiệp vụ của ứng dụng. Nó bao gồm các thành phần như dịch vụ, bộ điều khiển trường hợp sử dụng và DTO.
% Lớp miền : Lớp này chịu trách nhiệm xác định các quy tắc nghiệp vụ và đối tượng miền của ứng dụng. Nó bao gồm các thành phần như thực thể, đối tượng giá trị và dịch vụ miền.
% Lớp Persistence : Lớp này chịu trách nhiệm lưu trữ và truy xuất dữ liệu từ cơ sở dữ liệu. Nó bao gồm các thành phần như kho lưu trữ, DAO và khung ORM.
% Luồng dữ liệu giữa các lớp trong kiến ​​trúc phân lớp thường là một chiều, trong đó mỗi lớp chỉ giao tiếp với lớp ngay bên dưới hoặc bên trên nó.

% Khi nào nên sử dụng
% Kiến trúc phân lớp là một mẫu kiến ​​trúc phổ biến phù hợp trong nhiều tình huống. Nó đặc biệt hữu ích khi xây dựng các hệ thống lớn, phức tạp cần dễ dàng bảo trì và mở rộng.

% Một số tình huống trong đó kiến ​​trúc phân lớp có thể phù hợp bao gồm:

% Các ứng dụng doanh nghiệp quy mô lớn có nhiều mô-đun hoặc hệ thống con khác nhau.
% Các ứng dụng cần hỗ trợ nhiều giao diện, chẳng hạn như web và di động.
% Các ứng dụng đòi hỏi mức độ linh hoạt và mô-đun cao, trong đó các thành phần riêng lẻ có thể dễ dàng hoán đổi hoặc thay thế.
% Các ứng dụng yêu cầu phân tách chặt chẽ các mối quan tâm, có ranh giới rõ ràng giữa các lớp khác nhau của hệ thống.
% Nói chung, kiến ​​trúc phân lớp có thể là một lựa chọn tốt khi bạn cần xây dựng một hệ thống dễ hiểu, dễ bảo trì và mở rộng theo thời gian. Nó có thể giúp bạn giữ mã của mình ngăn nắp và đảm bảo rằng mỗi lớp của hệ thống đều có trách nhiệm được xác định rõ ràng.

% Khi nào nên tránh
% Kiến trúc phân lớp có thể không phù hợp trong một số trường hợp nhất định, bao gồm:

% Dự án nhỏ: Đối với các dự án nhỏ có yêu cầu đơn giản, kiến ​​trúc phân lớp có thể gây ra sự phức tạp không cần thiết.
% Các dự án có thay đổi thường xuyên: Nếu một dự án yêu cầu thay đổi thường xuyên, kiến ​​trúc phân lớp có thể gây khó khăn cho việc sửa đổi ứng dụng vì những thay đổi đối với một lớp có thể ảnh hưởng đến các lớp khác.
% Các dự án có yêu cầu thay đổi: Nếu các yêu cầu của dự án không được hiểu rõ hoặc có khả năng thay đổi thường xuyên thì kiến ​​trúc phân lớp có thể không phải là cách tiếp cận tốt nhất vì khó có thể sửa đổi ứng dụng để phù hợp với các yêu cầu thay đổi.
% Các dự án yêu cầu hiệu suất cao: Kiến trúc phân lớp có thể gây ra chi phí về thời gian xử lý và mức sử dụng bộ nhớ, điều này có thể ảnh hưởng đến hiệu suất của ứng dụng.
% Dự án có độ phức tạp thấp: Đối với các ứng dụng có độ phức tạp thấp, kiến ​​trúc phân lớp có thể là quá mức cần thiết và có thể không mang lại bất kỳ lợi ích đáng kể nào so với kiến ​​trúc đơn giản hơn.
% So sánh với kiến ​​trúc lát dọc
% Kiến trúc phân lớp và kiến ​​trúc lát dọc đều được sử dụng phổ biến trong phát triển phần mềm, nhưng chúng khác nhau về cách tiếp cận cấu trúc mã.

% Kiến trúc phân lớp dựa trên sự phân tách chặt chẽ các mối quan tâm, trong đó mỗi lớp có trách nhiệm cụ thể và chỉ giao tiếp với các lớp liền kề. Thông thường, các lớp trong kiến ​​trúc phân lớp bao gồm lớp trình bày, lớp ứng dụng, lớp miền và lớp dữ liệu. Kiến trúc này rất phù hợp cho các hệ thống lớn, phức tạp, trong đó việc duy trì sự tách biệt giữa các thành phần là rất quan trọng.

% Mặt khác, kiến ​​trúc lát cắt dọc tập trung vào việc tổ chức mã xung quanh các tính năng hoặc chức năng cụ thể, với tất cả các lớp được biểu thị trong mỗi lát cắt dọc. Trong kiến ​​trúc này, mỗi lát bao gồm một lớp trình bày, lớp ứng dụng, lớp miền và lớp dữ liệu, tất cả đều tập trung vào một tính năng hoặc chức năng cụ thể. Kiến trúc này rất phù hợp cho các hệ thống nhỏ hơn, tập trung hơn và cho các nhóm ưu tiên phát triển nhanh chóng và triển khai thường xuyên.

% Nhìn chung, kiến ​​trúc phân lớp phù hợp hơn với các hệ thống lớn, phức tạp, trong đó khả năng bảo trì và mở rộng là quan trọng, trong khi kiến ​​trúc lát dọc phù hợp hơn với các hệ thống nhỏ hơn, tập trung hơn, nơi việc phát triển nhanh chóng và triển khai thường xuyên là quan trọng.


<!--@\07DomainDrivenDesignTacticalPatterns_VVN\000000013.srt-->

<!--Dịch vụ ứng dụng (app sẻvice)-->

Chúng ta hãy xem lại định nghĩa về dịch vụ miền . Nó tuyên bố rằng dịch vụ miền là một đối tượng miền thực hiện chức năng miền.

Và vì dịch vụ danh mục khách hàng sẽ không triển khai bất kỳ chức năng miền nào nên chúng tôi không thể triển khai nó dưới dạng dịch vụ miền.

Và đây là nơi các dịch vụ ứng dụng xuất hiện. Đó là một định nghĩa chính thức hơn về một dịch vụ ứng dụng.

Nó là một đối tượng miền không triển khai bất kỳ chức năng miền nào mà phụ thuộc vào các đối tượng miền khác để hiển thị chức năng miền cấp cao cho bên ngoài của người tiêu dùng đối với mô hình.

Sự khác biệt chính giữa dịch vụ miền và dịch vụ ứng dụng là dịch vụ ứng dụng không triển khai bất kỳ loại logic nghiệp vụ hoặc chức năng miền nào.

Sự khác biệt lớn khác là dịch vụ ứng dụng được tiếp xúc với người tiêu dùng bên ngoài như ứng dụng Web, ứng dụng di động hoặc dịch vụ ứng dụng.

Chúng ta hãy đi qua các đặc điểm của một dịch vụ ứng dụng. Dịch vụ ứng dụng không có logic miền và đây là điểm khác biệt chính giữa dịch vụ ứng dụng và dịch vụ miền.

Các dịch vụ ứng dụng như dịch vụ miền đều không có trạng thái. Các dịch vụ ứng dụng có thể xác định giao diện bên ngoài, các dịch vụ ứng dụng được hiển thị hoặc một số loại giao thức mạng.

Chúng ta hãy đi qua các chi tiết của từng trong số này. Một dịch vụ ứng dụng không có logic miền. Nó phụ thuộc vào đối tượng miền khác cho logic miền.

Đây là điểm khác biệt chính giữa dịch vụ miền và dịch vụ ứng dụng. Dịch vụ ứng dụng điều phối việc thực thi logic miền.

Giống như dịch vụ miền và dịch vụ ứng dụng cũng không có trạng thái. Không có quản lý nhà nước được thực hiện trong dịch vụ ứng dụng.

Không có biến trạng thái hoặc sự tồn tại lâu dài của các đối tượng miền được triển khai trong dịch vụ ứng dụng. Dịch vụ ứng dụng phụ thuộc vào đối tượng miền để tồn tại lâu dài và dịch vụ ứng dụng hiển thị giao diện được thế giới bên ngoài sử dụng.

Nói cách khác, lược đồ yêu cầu và phản hồi cho dịch vụ ứng dụng không cần phải liên kết với bất kỳ đối tượng miền nào khác.

Dịch vụ ứng dụng hiển thị giao diện bên ngoài hoặc giao thức mạng trong mô hình miền. Dịch vụ ứng dụng có thể được coi như một đối tượng ranh giới bảo vệ tất cả các đối tượng trong mô hình miền.

Dịch vụ ứng dụng có thể được hiển thị dưới dạng API và API này được các thành phần bên ngoài sử dụng qua giao thức mạng.

Giao thức mạng này, có thể là SCDP, MQ hoặc thậm chí có thể là giao thức độc quyền. Định dạng dữ liệu giữa năng lực bên ngoài và API rất linh hoạt.

Nó có thể là Jason Ximo, CSFI hoặc bất kỳ định dạng nào khác. Tùy thuộc vào việc thực hiện dịch vụ ứng dụng.

Các thành phần bên ngoài có thể có hoặc không có kiến ​​thức về đối tượng miền hoặc cấu trúc của chúng. Tiếp theo, tôi sẽ thảo luận về mối quan hệ giữa dịch vụ ứng dụng và dịch vụ miền và dịch vụ ứng dụng có thể hiển thị dịch vụ miền với thành phần bên ngoài.

Dịch vụ miền để cung cấp giao diện cho các thành phần bên ngoài. Đã đến lúc đi vào những điểm chính trong bài học này chúng ta đã học về các ứng dụng, dịch vụ, ứng dụng, dịch vụ không triển khai bất kỳ hành vi miền nào.

Chúng cung cấp các dịch vụ cấp cao bằng cách phối hợp thực thi logic miền trong các đối tượng miền.

Các dịch vụ ứng dụng hiển thị giao diện cho các thành phần bên ngoài. Nghĩa là, các thành phần nằm ngoài mô hình miền thông qua giao thức mạng như HTP và NQ.













<!--@\07DomainDrivenDesignTacticalPatterns_VVN\000000014.srt-->

<!--Dịch vụ cơ sở hạ tầng-->

là dịch vụ tương tác với tài nguyên bên ngoài để giải quyết một vấn đề mối quan tâm không thuộc phạm vi vấn đề chính.

Nó xác định một hợp đồng được các đối tượng miền sử dụng để tương tác với các dịch vụ bên ngoài. Từ khóa ở đây là nguồn lực bên ngoài.

VD:

<!--Logging system e.g., Fluentd, ElastiSearch-->

<!--Ví dụ: thông báo qua email hoặc SMS-->

<!--CSDL bên ngoài hoặc thậm chí là hệ thống tệp-->

<!--Google Map.-->

Dịch vụ cơ sở hạ tầng không có logic miền.

Dịch vụ cơ sở hạ tầng tuân theo nguyên tắc trách nhiệm duy nhất

8

00: 01: 39, 420--> 00: 01: 50, 760

<!--Chúng ta hãy đi qua các chi tiết của từng một trong số này. Dịch vụ cơ sở hạ tầng không có logic miền vì nó cung cấp, như tên cho thấy, dịch vụ cơ sở hạ tầng chứ không phải dịch vụ kinh doanh.-->

9

00: 01: 50, 970--> 00: 02: 08, 000

<!--Nó không có bất kỳ sự phụ thuộc trực tiếp nào vào đối tượng miền và dịch vụ cơ sở hạ tầng được đối tượng miền và các dịch vụ sử dụng để tương tác với các tài nguyên bên ngoài và dịch vụ cơ sở hạ tầng tuân theo nguyên tắc trách nhiệm duy nhất.-->

10

00: 02: 08, 040--> 00: 02: 18, 640

<!--Ý tưởng là dịch vụ này cung cấp chức năng cho một và chỉ một thứ. Mục đích của họ là đơn giản hóa việc triển khai và làm cho dịch vụ trở nên dễ hiểu.-->

11

00: 02: 18, 660--> 00: 02: 29, 100

<!--Ví dụ: chúng tôi có ba dịch vụ này, mỗi dịch vụ chuyên cung cấp một chức năng cụ thể. Ví dụ: dịch vụ email chỉ để gửi email.-->

12

00: 02: 29, 130--> 00: 02: 38, 520

<!--Dịch vụ ghi nhật ký chỉ để ghi nhật ký tin nhắn và dịch vụ CSDL là để tương tác với CSDL và cơ sở hạ tầng.-->

13

00: 02: 38, 520--> 00: 02: 52, 140

<!--Dịch vụ xác định một hợp đồng tiêu chuẩn giữa mô hình và các tài nguyên bên ngoài. Hãy nghĩ về nó giống như một API, dành cho các đối tượng và dịch vụ mô hình sử dụng.-->

14

00: 02: 52, 620--> 00: 03: 03, 990

<!--Và nó cũng sẽ thực hiện bất kỳ loại chuyển đổi nào cần thiết trên dữ liệu. Bây giờ hãy xem cơ chế này làm cho miền độc lập hơn với tài nguyên bên ngoài như thế nào.-->

<!--Giả sử chúng ta phải triển khai một dịch vụ email. Dịch vụ e-mail này sẽ cung cấp chức năng tiêu chuẩn để gửi email.-->

<!--Ban đầu, dịch vụ e-mail được triển khai bằng cách sử dụng sendmail của Linux. Nhưng giả sử trong một khoảng thời gian, số lượng email được gửi đi từ ứng dụng tăng lên và do đó cần có một giải pháp mạnh mẽ hơn và Sendmail đã được thay thế bằng MailChimp.-->

Thay đổi này sẽ chỉ yêu cầu thay đổi trong dịch vụ email và sẽ không có tác động đến bất kỳ dịch vụ miền nào sử dụng dịch vụ email nội dung hiển thị theo hợp đồng tiêu chuẩn và do đó mô hình miền được cách ly khỏi các thay đổi tài nguyên bên ngoài.

Trong bài giảng này, chúng ta đã tìm hiểu về các dịch vụ cơ sở hạ tầng. Các dịch vụ cơ sở hạ tầng như dịch vụ ứng dụng không thực hiện bất kỳ hành vi miền nào.

Các dịch vụ cơ sở hạ tầng cung cấp các tài nguyên bên ngoài thông qua giao diện tiêu chuẩn hoặc hợp đồng tiêu chuẩn và cơ chế hợp đồng tiêu chuẩn này bảo vệ mô hình miền khỏi những thay đổi trong dịch vụ bên ngoài.

<!--Hướng dẫn 7/15-->

<!--Hướng dẫn 7/16-->

<!--@ Xong DDD -->

<!--@ Xong DDD -->

<!--@ Xong DDD -->

<!--@ Xong DDD -->
