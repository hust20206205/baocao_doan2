Để đơn giản bài toán, các chức năng trong đồ án này đã thay đổi so với bài toán thực tế như sau:

\begin{itemize}

\item Bỏ qua định dạng tập tin của hóa đơn

\begin{itemize}

\item Trong đồ án này, em bỏ qua định dạng tập tin của hóa đơn ví dụ: XML, PDF, EXCEL, \dots

\end{itemize}

\item Thay đổi phần mã số thuế và chữ ký số

\begin{itemize}

\item Mã số thuế được tạo khi sau đăng ký phần mềm quản lý hóa đơn điện tử.

\begin{itemize}

\item Trong thực tế, cơ quan thuế quản lý mã số thuế có 10 ký tự đại diện cá nhân, doanh nghiệp hoặc 14 ký tự đại diện chi nhánh của doanh nghiệp.

% Người nộp thuế dùng mã số thuế gửi đăng ký tới Tổng cục Thuế theo 2 cách:

% \item Trong thực tế, cơ quan thuế quản lý mã số thuế có 10 ký tự đại diện cá nhân, doanh nghiệp hoặc 14 ký tự đại diện chi nhánh của doanh nghiệp. Người nộp thuế dùng mã số thuế gửi đăng ký tới Tổng cục Thuế theo 2 cách:

% \begin{itemize}

% \item Đăng ký qua tổ chức cung cấp dịch vụ quản lý hóa đơn điện tử.

% \item Đăng ký trên cổng thông tin điện tử của Tổng cục Thuế.

% \end{itemize}

\item Trong đồ án này, khi người nộp thuế đăng ký tài khoản của phần mềm quản lý hóa đơn điện tử, mã số thuế được tạo bằng cách phần mềm quản lí hóa đơn gửi yêu cầu đến tct-demo.

\end{itemize}

\item Bỏ qua phần chữ ký số.

% thay thế bằng ứng dụng xác thực.

\begin{itemize}

\item Trong thực tế, chữ ký số của mọi doanh nghiệp là USB Token. Trong đồ án này, em xin phép bỏ qua phần chữ ký số.

% thay thế chữ ký số bằng ứng dụng xác thực 2 yếu tố sau khi đăng nhập phần mềm quản lý hóa đơn điện tử.

% \item Trong đồ án này, em thay thế chữ ký số bằng ứng dụng xác thực 2 yếu tố sau khi đăng nhập phần mềm quản lý hóa đơn điện tử.

% \begin{example} Ứng dụng xác thực như Google Authenticator, Microsoft Authenticator, \dots

% \end{example}

\end{itemize}

\end{itemize}

% \item Bỏ qua phần ký hiệu hóa đơn

% % https://www.meinvoice.vn/tin-tuc/12961/mau-so-hoa-don-va-ky-hieu-hoa-don-dien-tu-nd123-tt78

\item Bỏ qua chức năng lập hóa đơn điều chỉnh

\begin{itemize}

\item Trong đồ án này, em bỏ qua chức năng lập hóa đơn điều chỉnh và chỉ có chức năng lập hóa đơn thay thế (sửa hóa đơn).

\end{itemize}

\item Bỏ qua chức năng phê duyệt hóa đơn

\item Bỏ qua chức năng đa người thuê (multi-tenancy)

\begin{itemize}

\item Trong đồ án này, em bỏ qua chức năng đa người thuê, đối với 1 người nộp thuế chỉ có 1 tài khoản tương tác quản lý hóa đơn điện tử.

% (multi-tenancy) và chỉ có chức năng đơn người thuê (single-tenancy).

\end{itemize}

\end{itemize}