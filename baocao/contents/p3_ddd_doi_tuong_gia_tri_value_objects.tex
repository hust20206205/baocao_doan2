Đối tượng giá trị (Value Objects) là đối tượng đại diện cho một đặc điểm của miền mà không có nhận dạng riêng. Một điểm khác biệt quan trọng giữa thực thể và đối tượng giá trị là đối tượng giá trị không tồn tại lâu dài trong CSDL. Đối tượng giá trị được tạo trong bộ nhớ tiến trình và sau đó bị hủy sau khi nó đã phục vụ mục đích của nó. Các đối tượng giá trị được so sánh dựa trên thuộc tính không phải danh tính của chúng, điều này cho phép so sánh linh hoạt và trực quan hơn.

Đối tượng giá trị trong một bối cảnh bị giới hạn này có thể trở thành một thực thể trong bối cảnh bị giới hạn khác. Sự khác biệt giữa các đối tượng giá trị và các thực thể phụ thuộc vào ngữ cảnh và dựa trên nhu cầu của mô hình miền. Vì vậy, việc lựa chọn mô hình hóa một đối tượng là đối tượng giá trị hay thực thể phụ thuộc vào nhu cầu của mô hình miền và các yêu cầu cụ thể của giải pháp phần mềm đang được phát triển. Điều quan trọng là chọn các khái niệm mô hình phù hợp để thể hiện chính xác miền và cung cấp giải pháp phần mềm linh hoạt và có thể bảo trì.

\begin{example} Giá tiền là đối tượng giá trị vì nó có thể được biểu diễn dưới dạng các thuộc tính. \end{example}

% #

% #

% #

% #

% #

% #

% #

%! VD

%! chúng ta sẽ đặt logic xác thực cho địa chỉ email ở đâu?

%! xác nhận kỹ thuật không liên quan đến bất kỳ khái niệm kinh doanh nào.

%! tạo một đối tượng giá trị để xác thực địa chỉ email.

%! Kết quả là, thực thể khách hàng sẽ sạch hơn và đơn giản hơn nhiều trong việc thực hiện.

% Một ví dụ

% Giả sử chúng ta đang xây dựng một trang web thương mại điện tử và chúng ta có yêu cầu thể hiện giá trong mô hình miền. Giá là một ứng cử viên phù hợp cho đối tượng giá trị vì nó có thể được biểu diễn dưới dạng kết hợp các thuộc tính (ví dụ: tiền tệ và số lượng) và không có đặc điểm nhận dạng duy nhất của riêng nó.

% % Đối tượng giá trị so với thực thể

% Đối tượng giá trị trong một giải pháp có thể trở thành một thực thể trong giải pháp khác. Sự khác biệt giữa các đối tượng giá trị và các thực thể phụ thuộc vào ngữ cảnh và dựa trên nhu cầu của mô hình miền.

% Trong một ngữ cảnh, một đối tượng có thể được coi là một đối tượng giá trị vì nó được xác định bởi các thuộc tính của nó và không có một danh tính duy nhất. Tuy nhiên, trong một bối cảnh khác hoặc một giải pháp khác, cùng một đối tượng có thể được coi là một thực thể vì nó có một danh tính duy nhất và có thể được lưu giữ trong CSDL hoặc có tuổi thọ vượt quá đối tượng chứa.

% % Ví dụ: hãy xem xét một đối tượng Person trong miền nhân sự. Trong một ngữ cảnh, đối tượng Person có thể được coi là một đối tượng giá trị vì nó được xác định bởi các thuộc tính của nó, chẳng hạn như tên, ngày sinh và số an sinh xã hội. Tuy nhiên, trong một ngữ cảnh khác hoặc một giải pháp khác, đối tượng Person có thể được coi là một thực thể vì nó có một danh tính duy nhất và có thể được duy trì trong CSDL hoặc được sử dụng trên các ngữ cảnh giới hạn khác nhau.

% Cuối cùng, việc lựa chọn mô hình hóa một đối tượng là đối tượng giá trị hay thực thể phụ thuộc vào nhu cầu của mô hình miền và các yêu cầu cụ thể của giải pháp phần mềm đang được phát triển. Điều quan trọng là chọn các khái niệm mô hình phù hợp để thể hiện chính xác miền và cung cấp giải pháp phần mềm linh hoạt và có thể bảo trì.