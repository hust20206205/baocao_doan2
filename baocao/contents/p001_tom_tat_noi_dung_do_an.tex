Báo cáo đồ án này được tổ chức thành các phần chính sau:

#

\begin{itemize}

\item \textbf{Chương 1: xxxxxxxxxxxxxxxxx}

\begin{quote}

xxxxxxxxxxxxxxxxx

\end{quote}

\item \textbf{Chương 1: xxxxxxxxxxxxxxxxx}

\begin{quote}

xxxxxxxxxxxxxxxxx

\end{quote}

\item \textbf{Chương 1: xxxxxxxxxxxxxxxxx}

\begin{quote}

xxxxxxxxxxxxxxxxx

\end{quote}

\end{itemize}

%@ Thêm các mục nhỏ như:

%@ Thêm các mục nhỏ như:

%@ Thêm các mục nhỏ như:

%@ Thêm các mục nhỏ như:

%@ Thêm các mục nhỏ như:

%@ Thêm các mục nhỏ như:

%@ Thêm các mục nhỏ như:

%@ Thêm các mục nhỏ như:

%@ Thêm các mục nhỏ như:

%@ Thêm các mục nhỏ như:

%@ Thêm các mục nhỏ như:

%@ Thêm các mục nhỏ như:

%@ Thêm các mục nhỏ như:

% Luận văn được tổ chức thành các phần chính sau:

% Mở đầu: Trình bày tổng quan về đề tài

% Chương 1: Trình bày cách thức phát triển phần mềm theo kiến trúc kiến trúc vi dịch vụ .

% Trong chương này, luận văn tập trung làm rõ các nội dung:

% - Sơ lược về một số hướng kiến trúc phần mềm truyền thống như kiến trúc nguyên

% khối, kiến trúc hướng dịch vụ, công nghệ ESB

% - Tổng quan về kiến trúc kiến trúc vi dịch vụ : sự ra đời, đặc điểm của kiến trúc vi dịch vụ

% - Các mẫu thiết kế quan trọng được sử dụng trong kiến trúc vi dịch vụ

% - Một số nguyên tắc thiết kế kiến trúc vi dịch vụ

% Chương 2: Trình bày hướng xây dựng ứng dụng web sử dụng micro - frontends.

% Trong chương này, luận văn tập trung làm rõ các nội dung:

% - Sơ lược về một số mô hình phát triển web như mô hình web tĩnh, mô hình web

% động, mô hình web theo hướng SPA

% - Sự ra đời của kiến trúc micro - frontends

% - Các cơ chế tích hợp micro - frontends được thảo luận như: tích hợp theo hướng

% “build - time”, tích hợp theo hướng “run - time”, cách thức điều hướng và giao tiếp

% giữa các micro - frontends

% Chương 3: Trình bày cách thức xây dựng một ứng dụng thử nghiệm sử dụng kiến

% trúc kiến trúc vi dịch vụ, micro - frontends. Một số nội dung chính trong quá trình thực nghiệm

% được làm rõ bao gồm:

% 3

% - Áp dụng phương pháp thiết kế hướng miền để phân hoạch, thiết kế chương trình

% - Thiết kế và cài đặt tầng dịch vụ theo hướng kiến trúc vi dịch vụ, sử dụng các công

% nghệ trên nền tảng Java như Spring Boot, Spring Cloud

% - Thiết kế và cài đặt tầng giao diện theo hướng micro - frontends, sử dụng các công

% nghệ như Single - SPA, Angular, ReactJS

% - Một số kỹ thuật kiểm thử kiến trúc vi dịch vụ cũng được thảo luận như kiểm thử đơn

% vị, kiểm thử tích hợp và kiểm thử mức giao diện

% - Cách thức triển khai ứng dụng sử dụng Docker

% Phần kết luận: Tổng kết, đánh giá kết quả thu được của quá trình nghiên cứu cũng

% như các ưu nhược điểm, các hạn chế và hướng phát triển tương lai.