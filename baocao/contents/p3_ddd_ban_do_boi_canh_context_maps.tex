Những mối quan hệ của bối cảnh bị giới hạn cần được quản lý chặt chẽ để hoạt động độc lập, nhất quán và linh hoạt. Do đó, chúng ta cần phải ghi lại các mối quan hệ thông qua việc sử dụng bản đồ bối cảnh. \emph{Bản đồ bối cảnh (Context Maps)} là sự thể hiện trực quan của hệ thống, thể hiện các thành phần và mối quan hệ giữa các thành phần.

\begin{figure}[H]

    \centering

    \includegraphics[scale = 0.4]{pictures/_vi_du_ban_do_boi_canh_trong_1_ngan_hang/main.drawio.png}

    \caption{Ví dụ bản đồ bối cảnh trong 1 ngân hàng}

\end{figure}

\subsubsection{Áp dụng bản đồ bối cảnh}    


\begin{itemize}
    \item   Do user - service    và invoice     -      service  cần sử dụng uuid của tct     -     demo nên tct     -     demo là OHS và user     -      service, invoice     -      service cần tuân thủ.


    \item    Tiếp theo, user     -      service    cung cấp  invoice     -      service, và invoice     -      service      cung cấp  report     -      service   do sử dụng kiến trúc lục giác (Hexagonal      Architecture) nên  bối cảnh bị giới hạn hạ lưu       là ACL.
\end{itemize}







\begin{figure}[H]

    \centering

    \includegraphics[scale = 0.4]{pictures/_ap_dung_ban_do_boi_canh/freelancer.main.excalidraw.png}

    \caption{Áp dụng bản đồ bối cảnh trong đồ án}

\end{figure}